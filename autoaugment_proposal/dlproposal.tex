\documentclass[10pt,twocolumn,letterpaper]{article}

\usepackage{cvpr}
\usepackage{times}
\usepackage{epsfig}
\usepackage{graphicx}
\usepackage{amsmath}
\usepackage{amssymb}

% Include other packages here, before hyperref.

% If you comment hyperref and then uncomment it, you should delete
% egpaper.aux before re-running latex.  (Or just hit 'q' on the first latex
% run, let it finish, and you should be clear).
\usepackage[breaklinks=true,bookmarks=false]{hyperref}

\cvprfinalcopy % *** Uncomment this line for the final submission

\def\cvprPaperID{****} % *** Enter the CVPR Paper ID here
\def\httilde{\mbox{\tt\raisebox{-.5ex}{\symbol{126}}}}

% Pages are numbered in submission mode, and unnumbered in camera-ready
%\ifcvprfinal\pagestyle{empty}\fi
\setcounter{page}{1}
\begin{document}

%%%%%%%%% TITLE
\title{Proposal: An Analysis of Learning Augmentation Policies from Data}

\author{
    	\small{NAMES: Samuel, Gian K\"onig, Colin K\"alin} \\
   	\small{NETHZ: fsamuel, koenigg, ckaelin}\\
	\small{EMAIL: \{fsamuel, koenigg, ckaelin\}$@$student.ethz.ch}\\
    	\small{ID: XX-XXX-XXX, 09-913-245, XX-XXX-XXX}
}

\maketitle
%\thispagestyle{empty}

%%%%%%%%% ABSTRACT
\begin{abstract}
   The ABSTRACT~\cite{DBLP:journals/corr/abs-1805-09501}\footnote {This is what a footnote looks like.  It
often distracts the reader from the main flow of the argument.} is to be in fully-justified italicized text, at the top
   of the left-hand column, below the author and affiliation
   information. Use the word ``Abstract'' as the title, in 12-point
   Times, boldface type, centered relative to the column, initially
   capitalized. The abstract is to be in 10-point, single-spaced type.
   Leave two blank lines after the Abstract, then begin the main text.
   Look at previous CVPR abstracts to get a feel for style and length.
\end{abstract}

%%%%%%%%% BODY TEXT
\section{Introduction}

Please follow the steps outlined below when submitting your manuscript to
the IEEE Computer Society Press.  This style guide now has several
important modifications (for example, you are no longer warned against the
use of sticky tape to attach your artwork to the paper), so all authors
should read this new version.

\section{What has been done}

Summary of ~\cite{DBLP:journals/corr/abs-1805-09501}

\section{Problems we want to investigate}

{\small
\bibliographystyle{ieee}
\bibliography{dlbib}
}

\end{document}
